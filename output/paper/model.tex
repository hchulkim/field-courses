% Options for packages loaded elsewhere
\PassOptionsToPackage{unicode}{hyperref}
\PassOptionsToPackage{hyphens}{url}
\PassOptionsToPackage{dvipsnames,svgnames,x11names}{xcolor}
%
\documentclass[
  letterpaper,
  DIV=11,
  numbers=noendperiod]{scrartcl}

\usepackage{amsmath,amssymb}
\usepackage{iftex}
\ifPDFTeX
  \usepackage[T1]{fontenc}
  \usepackage[utf8]{inputenc}
  \usepackage{textcomp} % provide euro and other symbols
\else % if luatex or xetex
  \usepackage{unicode-math}
  \defaultfontfeatures{Scale=MatchLowercase}
  \defaultfontfeatures[\rmfamily]{Ligatures=TeX,Scale=1}
\fi
\usepackage{lmodern}
\ifPDFTeX\else  
    % xetex/luatex font selection
\fi
% Use upquote if available, for straight quotes in verbatim environments
\IfFileExists{upquote.sty}{\usepackage{upquote}}{}
\IfFileExists{microtype.sty}{% use microtype if available
  \usepackage[]{microtype}
  \UseMicrotypeSet[protrusion]{basicmath} % disable protrusion for tt fonts
}{}
\makeatletter
\@ifundefined{KOMAClassName}{% if non-KOMA class
  \IfFileExists{parskip.sty}{%
    \usepackage{parskip}
  }{% else
    \setlength{\parindent}{0pt}
    \setlength{\parskip}{6pt plus 2pt minus 1pt}}
}{% if KOMA class
  \KOMAoptions{parskip=half}}
\makeatother
\usepackage{xcolor}
\setlength{\emergencystretch}{3em} % prevent overfull lines
\setcounter{secnumdepth}{-\maxdimen} % remove section numbering
% Make \paragraph and \subparagraph free-standing
\makeatletter
\ifx\paragraph\undefined\else
  \let\oldparagraph\paragraph
  \renewcommand{\paragraph}{
    \@ifstar
      \xxxParagraphStar
      \xxxParagraphNoStar
  }
  \newcommand{\xxxParagraphStar}[1]{\oldparagraph*{#1}\mbox{}}
  \newcommand{\xxxParagraphNoStar}[1]{\oldparagraph{#1}\mbox{}}
\fi
\ifx\subparagraph\undefined\else
  \let\oldsubparagraph\subparagraph
  \renewcommand{\subparagraph}{
    \@ifstar
      \xxxSubParagraphStar
      \xxxSubParagraphNoStar
  }
  \newcommand{\xxxSubParagraphStar}[1]{\oldsubparagraph*{#1}\mbox{}}
  \newcommand{\xxxSubParagraphNoStar}[1]{\oldsubparagraph{#1}\mbox{}}
\fi
\makeatother


\providecommand{\tightlist}{%
  \setlength{\itemsep}{0pt}\setlength{\parskip}{0pt}}\usepackage{longtable,booktabs,array}
\usepackage{calc} % for calculating minipage widths
% Correct order of tables after \paragraph or \subparagraph
\usepackage{etoolbox}
\makeatletter
\patchcmd\longtable{\par}{\if@noskipsec\mbox{}\fi\par}{}{}
\makeatother
% Allow footnotes in longtable head/foot
\IfFileExists{footnotehyper.sty}{\usepackage{footnotehyper}}{\usepackage{footnote}}
\makesavenoteenv{longtable}
\usepackage{graphicx}
\makeatletter
\newsavebox\pandoc@box
\newcommand*\pandocbounded[1]{% scales image to fit in text height/width
  \sbox\pandoc@box{#1}%
  \Gscale@div\@tempa{\textheight}{\dimexpr\ht\pandoc@box+\dp\pandoc@box\relax}%
  \Gscale@div\@tempb{\linewidth}{\wd\pandoc@box}%
  \ifdim\@tempb\p@<\@tempa\p@\let\@tempa\@tempb\fi% select the smaller of both
  \ifdim\@tempa\p@<\p@\scalebox{\@tempa}{\usebox\pandoc@box}%
  \else\usebox{\pandoc@box}%
  \fi%
}
% Set default figure placement to htbp
\def\fps@figure{htbp}
\makeatother

\usepackage{fvextra}
\DefineVerbatimEnvironment{Highlighting}{Verbatim}{breaklines,commandchars=\\\{\}}
\DefineVerbatimEnvironment{OutputCode}{Verbatim}{breaklines,commandchars=\\\{\}}
\usepackage{booktabs}
\usepackage{longtable}
\usepackage{array}
\usepackage{multirow}
\usepackage{wrapfig}
\usepackage{float}
\usepackage{colortbl}
\usepackage{pdflscape}
\usepackage{tabu}
\usepackage{threeparttable}
\usepackage{threeparttablex}
\usepackage[normalem]{ulem}
\usepackage{makecell}
\usepackage{xcolor}
\KOMAoption{captions}{tableheading}
\makeatletter
\@ifpackageloaded{caption}{}{\usepackage{caption}}
\AtBeginDocument{%
\ifdefined\contentsname
  \renewcommand*\contentsname{Table of contents}
\else
  \newcommand\contentsname{Table of contents}
\fi
\ifdefined\listfigurename
  \renewcommand*\listfigurename{List of Figures}
\else
  \newcommand\listfigurename{List of Figures}
\fi
\ifdefined\listtablename
  \renewcommand*\listtablename{List of Tables}
\else
  \newcommand\listtablename{List of Tables}
\fi
\ifdefined\figurename
  \renewcommand*\figurename{Figure}
\else
  \newcommand\figurename{Figure}
\fi
\ifdefined\tablename
  \renewcommand*\tablename{Table}
\else
  \newcommand\tablename{Table}
\fi
}
\@ifpackageloaded{float}{}{\usepackage{float}}
\floatstyle{ruled}
\@ifundefined{c@chapter}{\newfloat{codelisting}{h}{lop}}{\newfloat{codelisting}{h}{lop}[chapter]}
\floatname{codelisting}{Listing}
\newcommand*\listoflistings{\listof{codelisting}{List of Listings}}
\makeatother
\makeatletter
\makeatother
\makeatletter
\@ifpackageloaded{caption}{}{\usepackage{caption}}
\@ifpackageloaded{subcaption}{}{\usepackage{subcaption}}
\makeatother

\usepackage{bookmark}

\IfFileExists{xurl.sty}{\usepackage{xurl}}{} % add URL line breaks if available
\urlstyle{same} % disable monospaced font for URLs
\hypersetup{
  pdftitle={Derivation of sectoral gravity model},
  pdfauthor={Hyoungchul Kim},
  colorlinks=true,
  linkcolor={blue},
  filecolor={Maroon},
  citecolor={Blue},
  urlcolor={Blue},
  pdfcreator={LaTeX via pandoc}}


\title{Derivation of sectoral gravity model}
\author{Hyoungchul Kim}
\date{2025-11-14}

\begin{document}
\maketitle


\subsection{Solving within a sector}\label{solving-within-a-sector}

\begin{itemize}
\tightlist
\item
  Just to make this simple, I will supress the sector index \(l\) for
  now.
\end{itemize}

I solve the following maximization problem:

\[
  \max_{c_{ij}} \left\{ \sum_i^N \left( \beta_i \right)^{\frac{1-\sigma}{\sigma}} \left( c_{ij} \right)^{\frac{\sigma - 1}{\sigma}} \right\}^{\frac{\sigma}{\sigma - 1}}
\]

subject to:

\[
  \sum_{i=1}^N p_{ij} c_{ij} \leq Y_j.
\]

Solving the lagrangian gives the following first order conditions:

\[
  \frac{\partial L}{\partial c_{ij}} = \left( \beta_i \right)^{\frac{1-\sigma}{\sigma}} \left( c_{ij} \right)^{\frac{\sigma - 1}{\sigma} - 1} - \lambda p_{ij} = 0.
\]

Now divide this by \(i'\)'s FOC, we get:

\[
  \frac{p_{ij}}{p_{i'j}} = \left( \frac{\beta_i}{\beta_{i'}} \right)^{\frac{1-\sigma}{\sigma}} \cdot \left( \frac{c_{ij}}{c_{i'j}} \right)^{- \frac{1}{\sigma}}.
\]

I can then rearrange this to get the expresson for \(c_{i'j}\):

\[
  c_{i'j} = \left( \frac{\beta_{i}}{\beta_{i'}} \right)^{\sigma - 1} \left( \frac{p_{ij}}{p_{i'j}} \right)^{\sigma} \cdot c_{ij}.
\]

Now I can substitute this back into the budget constraint to get the
expression for \(c_{ij}\):

\[
  \sum_{i=1}^N p_{ij} \left( \left( \frac{\beta_{i}}{\beta_{i'}} \right)^{\sigma - 1} \left( \frac{p_{ij}}{p_{i'j}} \right)^{\sigma} \cdot c_{ij} \right) = Y_j.
\]

This then becomes:

\[
  c_{ij} = \left( p_{ij} \right)^{-\sigma} \cdot \left( \frac{\beta_i}{P_j} \right)^{1-\sigma} \cdot Y_j
\]

where \(P_j\) is the price index:
\(P_j = \left[ \sum_{i=1}^N \left( \beta_i p_{ij} \right)^{1-\sigma} \right]^{\frac{1}{1-\sigma}}\).

Multiplying it by \(p_{ij}\) will give us the nominal demand \(X_{ij}\):

\[
  X_{ij} = p_{ij} \cdot c_{ij} = \left( p_{ij} \right)^{1-\sigma} \cdot \left( \frac{\beta_i}{P_j} \right)^{1-\sigma} \cdot Y_j.
\]

Now I can substitute
\(p_{ij} = p_i t_{ij} (1+\tau_{ij})(1-z_i)(1-(1- \phi_i)s_i)\):\footnote{Remember
  that substitution would have also happened for the price index.}

\[
  X_{ij} = \left[ p_i t_{ij} (1+\tau_{ij})(1-z_i)(1-(1- \phi_i)s_i) \right]^{1-\sigma} \cdot \left( \frac{\beta_i}{P_j} \right)^{1-\sigma} \cdot Y_j.
\]

Now I impose market clearing condition \(\sum_{j=1}^N X_{ij} = Y_{i}\)
and substitute \(X_{ij}\) with the expression above:

\[
  Y_i = \sum_{j=1}^N \left[ p_i t_{ij} (1+\tau_{ij})(1-z_i)(1-(1- \phi_i)s_i) \right]^{1-\sigma} \cdot \left( \frac{\beta_i}{P_j} \right)^{1-\sigma} \cdot Y_j.
\]

From now on, let's denote \(Y_j\) as \(E_j\) (expenditure equals to
income) to avoid confusion. If I divide both sides by the total sectoral
income \(Y\) I get the following expression:

\[
  \frac{Y_i}{Y} = \sum_{j=1}^N \left[ p_i t_{ij} (1+\tau_{ij})(1-z_i)(1-(1- \phi_i)s_i) \right]^{1-\sigma} \cdot \left( \frac{\beta_i}{P_j} \right)^{1-\sigma} \cdot \frac{E_j}{Y}.
\]

Now I can define
\(\Pi_i^{1-\sigma} = \sum_{j}\left( \frac{ t_{ij} ( 1+ \tau_{ij})}{P_j} \right)^{1-\sigma} \cdot \frac{E_j}{Y}\).
Then the expression above becomes:

\[
  \frac{Y_i}{Y} = \left( \beta_i p_i (1-z_i)(1-(1- \phi_i)s_i) \Pi_i  \right)^{1-\sigma}, \forall i.  
\]

Solve for
\(\left( \beta_i p_i (1-z_i)(1-(1- \phi_i)s_i)\right)^{1-\sigma}\):

\[
  \left( \beta_i p_i (1-z_i)(1-(1- \phi_i)s_i)\right)^{1-\sigma} = \frac{Y_i / Y}{ \Pi_i^{1-\sigma}}.
\]

Now put substitute the LHS into the price index expression:

\[
  P_j = \left[ \sum_{i=1}^N \left( \beta_i p_i t_{ij} (1+ \tau_{ij}) (1-z_i)(1-(1- \phi_i)s_i) \right)^{1-\sigma} \right]^{\frac{1}{1-\sigma}}.
\]

I can then gets inward multilateral resistance terms:

\[
  P_j^{1-\sigma} = \sum_i \left( \frac{t_{ij} (1 + \tau_{ij})}{\Pi_i} \right)^{1-\sigma} \cdot \frac{Y_i}{Y}.
\]

Lastly, I add the income and expenditure definitions and market clearing
conditions. For this case, I add back the sector index \(l\):

\[
  E_j^l = \alpha_j^l Y_j = \alpha_j^l \sum_l Y_j^l, 
\]

\[
  p_j^l = \frac{\left( Y_j^l / Y^l \right)^{\frac{1}{1-\sigma}}}{\beta_j^l (1-z_j^l)(1-(1- \phi_j^l)s_j^l) \Pi_j^l}. 
\]

Then done! I have derived the sectoral gravity model.

\subsection{References}\label{references}




\end{document}
