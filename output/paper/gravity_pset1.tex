% Options for packages loaded elsewhere
% Options for packages loaded elsewhere
\PassOptionsToPackage{unicode}{hyperref}
\PassOptionsToPackage{hyphens}{url}
\PassOptionsToPackage{dvipsnames,svgnames,x11names}{xcolor}
%
\documentclass[
  letterpaper,
  DIV=11,
  numbers=noendperiod]{scrartcl}
\usepackage{xcolor}
\usepackage{amsmath,amssymb}
\setcounter{secnumdepth}{-\maxdimen} % remove section numbering
\usepackage{iftex}
\ifPDFTeX
  \usepackage[T1]{fontenc}
  \usepackage[utf8]{inputenc}
  \usepackage{textcomp} % provide euro and other symbols
\else % if luatex or xetex
  \usepackage{unicode-math} % this also loads fontspec
  \defaultfontfeatures{Scale=MatchLowercase}
  \defaultfontfeatures[\rmfamily]{Ligatures=TeX,Scale=1}
\fi
\usepackage{lmodern}
\ifPDFTeX\else
  % xetex/luatex font selection
\fi
% Use upquote if available, for straight quotes in verbatim environments
\IfFileExists{upquote.sty}{\usepackage{upquote}}{}
\IfFileExists{microtype.sty}{% use microtype if available
  \usepackage[]{microtype}
  \UseMicrotypeSet[protrusion]{basicmath} % disable protrusion for tt fonts
}{}
\makeatletter
\@ifundefined{KOMAClassName}{% if non-KOMA class
  \IfFileExists{parskip.sty}{%
    \usepackage{parskip}
  }{% else
    \setlength{\parindent}{0pt}
    \setlength{\parskip}{6pt plus 2pt minus 1pt}}
}{% if KOMA class
  \KOMAoptions{parskip=half}}
\makeatother
% Make \paragraph and \subparagraph free-standing
\makeatletter
\ifx\paragraph\undefined\else
  \let\oldparagraph\paragraph
  \renewcommand{\paragraph}{
    \@ifstar
      \xxxParagraphStar
      \xxxParagraphNoStar
  }
  \newcommand{\xxxParagraphStar}[1]{\oldparagraph*{#1}\mbox{}}
  \newcommand{\xxxParagraphNoStar}[1]{\oldparagraph{#1}\mbox{}}
\fi
\ifx\subparagraph\undefined\else
  \let\oldsubparagraph\subparagraph
  \renewcommand{\subparagraph}{
    \@ifstar
      \xxxSubParagraphStar
      \xxxSubParagraphNoStar
  }
  \newcommand{\xxxSubParagraphStar}[1]{\oldsubparagraph*{#1}\mbox{}}
  \newcommand{\xxxSubParagraphNoStar}[1]{\oldsubparagraph{#1}\mbox{}}
\fi
\makeatother


\usepackage{longtable,booktabs,array}
\usepackage{calc} % for calculating minipage widths
% Correct order of tables after \paragraph or \subparagraph
\usepackage{etoolbox}
\makeatletter
\patchcmd\longtable{\par}{\if@noskipsec\mbox{}\fi\par}{}{}
\makeatother
% Allow footnotes in longtable head/foot
\IfFileExists{footnotehyper.sty}{\usepackage{footnotehyper}}{\usepackage{footnote}}
\makesavenoteenv{longtable}
\usepackage{graphicx}
\makeatletter
\newsavebox\pandoc@box
\newcommand*\pandocbounded[1]{% scales image to fit in text height/width
  \sbox\pandoc@box{#1}%
  \Gscale@div\@tempa{\textheight}{\dimexpr\ht\pandoc@box+\dp\pandoc@box\relax}%
  \Gscale@div\@tempb{\linewidth}{\wd\pandoc@box}%
  \ifdim\@tempb\p@<\@tempa\p@\let\@tempa\@tempb\fi% select the smaller of both
  \ifdim\@tempa\p@<\p@\scalebox{\@tempa}{\usebox\pandoc@box}%
  \else\usebox{\pandoc@box}%
  \fi%
}
% Set default figure placement to htbp
\def\fps@figure{htbp}
\makeatother





\setlength{\emergencystretch}{3em} % prevent overfull lines

\providecommand{\tightlist}{%
  \setlength{\itemsep}{0pt}\setlength{\parskip}{0pt}}



 


\usepackage{fvextra}
\DefineVerbatimEnvironment{Highlighting}{Verbatim}{breaklines,commandchars=\\\{\}}
\DefineVerbatimEnvironment{OutputCode}{Verbatim}{breaklines,commandchars=\\\{\}}
\usepackage{booktabs}
\usepackage{longtable}
\usepackage{array}
\usepackage{multirow}
\usepackage{wrapfig}
\usepackage{float}
\usepackage{colortbl}
\usepackage{pdflscape}
\usepackage{tabu}
\usepackage{threeparttable}
\usepackage{threeparttablex}
\usepackage[normalem]{ulem}
\usepackage{makecell}
\usepackage{xcolor}
\KOMAoption{captions}{tableheading}
\makeatletter
\@ifpackageloaded{caption}{}{\usepackage{caption}}
\AtBeginDocument{%
\ifdefined\contentsname
  \renewcommand*\contentsname{Table of contents}
\else
  \newcommand\contentsname{Table of contents}
\fi
\ifdefined\listfigurename
  \renewcommand*\listfigurename{List of Figures}
\else
  \newcommand\listfigurename{List of Figures}
\fi
\ifdefined\listtablename
  \renewcommand*\listtablename{List of Tables}
\else
  \newcommand\listtablename{List of Tables}
\fi
\ifdefined\figurename
  \renewcommand*\figurename{Figure}
\else
  \newcommand\figurename{Figure}
\fi
\ifdefined\tablename
  \renewcommand*\tablename{Table}
\else
  \newcommand\tablename{Table}
\fi
}
\@ifpackageloaded{float}{}{\usepackage{float}}
\floatstyle{ruled}
\@ifundefined{c@chapter}{\newfloat{codelisting}{h}{lop}}{\newfloat{codelisting}{h}{lop}[chapter]}
\floatname{codelisting}{Listing}
\newcommand*\listoflistings{\listof{codelisting}{List of Listings}}
\makeatother
\makeatletter
\makeatother
\makeatletter
\@ifpackageloaded{caption}{}{\usepackage{caption}}
\@ifpackageloaded{subcaption}{}{\usepackage{subcaption}}
\makeatother
\usepackage{bookmark}
\IfFileExists{xurl.sty}{\usepackage{xurl}}{} % add URL line breaks if available
\urlstyle{same}
\hypersetup{
  pdftitle={Gravity problem set},
  pdfauthor={Hyoungchul Kim},
  colorlinks=true,
  linkcolor={blue},
  filecolor={Maroon},
  citecolor={Blue},
  urlcolor={Blue},
  pdfcreator={LaTeX via pandoc}}


\title{Gravity problem set}
\author{Hyoungchul Kim}
\date{2025-11-30}
\begin{document}
\maketitle


\section{Part 1 (Theory): Derivation of sectoral gravity
model}\label{part-1-theory-derivation-of-sectoral-gravity-model}

\subsection{Solving within a sector}\label{solving-within-a-sector}

\begin{itemize}
\tightlist
\item
  Just to make this simple, I will suppress the sector index \(l\) for
  now.
\end{itemize}

I solve the following maximization problem:

\[
  \max_{c_{ij}} \left\{ \sum_i^N \left( \beta_i \right)^{\frac{1-\sigma}{\sigma}} \left( c_{ij} \right)^{\frac{\sigma - 1}{\sigma}} \right\}^{\frac{\sigma}{\sigma - 1}}
\]

subject to:

\[
  \sum_{i=1}^N p_{ij} c_{ij} \leq Y_j.
\]

Solving the lagrangian gives the following first order conditions:

\[
  \frac{\partial L}{\partial c_{ij}} = \left( \beta_i \right)^{\frac{1-\sigma}{\sigma}} \left( c_{ij} \right)^{\frac{\sigma - 1}{\sigma} - 1} - \lambda p_{ij} = 0.
\]

Now, dividing this by \(i'\)'s FOC, we get:

\[
  \frac{p_{ij}}{p_{i'j}} = \left( \frac{\beta_i}{\beta_{i'}} \right)^{\frac{1-\sigma}{\sigma}} \cdot \left( \frac{c_{ij}}{c_{i'j}} \right)^{- \frac{1}{\sigma}}.
\]

I can then rearrange this to get the expression for \(c_{i'j}\):

\[
  c_{i'j} = \left( \frac{\beta_{i}}{\beta_{i'}} \right)^{\sigma - 1} \left( \frac{p_{ij}}{p_{i'j}} \right)^{\sigma} \cdot c_{ij}.
\]

Now I can substitute this back into the budget constraint to get the
expression for \(c_{ij}\):

\[
  \sum_{i=1}^N p_{ij} \left( \left( \frac{\beta_{i}}{\beta_{i'}} \right)^{\sigma - 1} \left( \frac{p_{ij}}{p_{i'j}} \right)^{\sigma} \cdot c_{ij} \right) = Y_j.
\]

This then becomes:

\[
  c_{ij} = \left( p_{ij} \right)^{-\sigma} \cdot \left( \frac{\beta_i}{P_j} \right)^{1-\sigma} \cdot Y_j
\]

where \(P_j\) is the price index:
\(P_j = \left[ \sum_{i=1}^N \left( \beta_i p_{ij} \right)^{1-\sigma} \right]^{\frac{1}{1-\sigma}}\).

Multiplying it by \(p_{ij}\) will give us the nominal demand \(X_{ij}\):

\[
  X_{ij} = p_{ij} \cdot c_{ij} = \left( p_{ij} \right)^{1-\sigma} \cdot \left( \frac{\beta_i}{P_j} \right)^{1-\sigma} \cdot Y_j.
\]

Now I can substitute
\(p_{ij} = p_i t_{ij} (1+\tau_{ij})(1-z_i)(1-(1- \phi_i)s_i)\):\footnote{Remember
  that substitution would have also happened for the price index.}

\[
  X_{ij} = \left[ p_i t_{ij} (1+\tau_{ij})(1-z_i)(1-(1- \phi_i)s_i) \right]^{1-\sigma} \cdot \left( \frac{\beta_i}{P_j} \right)^{1-\sigma} \cdot Y_j.
\]

Now I impose the market clearing condition
\(\sum_{j=1}^N X_{ij} = Y_{i}\) and substitute \(X_{ij}\) with the
expression above:

\[
  Y_i = \sum_{j=1}^N \left[ p_i t_{ij} (1+\tau_{ij})(1-z_i)(1-(1- \phi_i)s_i) \right]^{1-\sigma} \cdot \left( \frac{\beta_i}{P_j} \right)^{1-\sigma} \cdot Y_j.
\]

From now on, let's denote \(Y_j\) as \(E_j\) (expenditure equals income)
to avoid confusion. If I divide both sides by the total sectoral income
\(Y\) I get the following expression:

\[
  \frac{Y_i}{Y} = \sum_{j=1}^N \left[ p_i t_{ij} (1+\tau_{ij})(1-z_i)(1-(1- \phi_i)s_i) \right]^{1-\sigma} \cdot \left( \frac{\beta_i}{P_j} \right)^{1-\sigma} \cdot \frac{E_j}{Y}.
\]

Now I can define
\(\Pi_i^{1-\sigma} = \sum_{j}\left( \frac{ t_{ij} ( 1+ \tau_{ij})}{P_j} \right)^{1-\sigma} \cdot \frac{E_j}{Y}\).
Then the expression above becomes:

\[
  \frac{Y_i}{Y} = \left( \beta_i p_i (1-z_i)(1-(1- \phi_i)s_i) \Pi_i  \right)^{1-\sigma}, \forall i.  
\]

Solving for
\(\left( \beta_i p_i (1-z_i)(1-(1- \phi_i)s_i)\right)^{1-\sigma}\):

\[
  \left( \beta_i p_i (1-z_i)(1-(1- \phi_i)s_i)\right)^{1-\sigma} = \frac{Y_i / Y}{ \Pi_i^{1-\sigma}}.
\]

Now, putting the LHS into the price index expression:

\[
  P_j = \left[ \sum_{i=1}^N \left( \beta_i p_i t_{ij} (1+ \tau_{ij}) (1-z_i)(1-(1- \phi_i)s_i) \right)^{1-\sigma} \right]^{\frac{1}{1-\sigma}}.
\]

I can then get inward multilateral resistance terms:

\[
  P_j^{1-\sigma} = \sum_i \left( \frac{t_{ij} (1 + \tau_{ij})}{\Pi_i} \right)^{1-\sigma} \cdot \frac{Y_i}{Y}.
\]

Lastly, I add the income and expenditure definitions and market clearing
conditions. For this case, I add back the sector index \(l\):

\[
  E_j^l = \alpha_j^l Y_j = \alpha_j^l \sum_l Y_j^l, 
\]

\[
  p_j^l = \frac{\left( Y_j^l / Y^l \right)^{\frac{1}{1-\sigma}}}{\beta_j^l (1-z_j^l)(1-(1- \phi_j^l)s_j^l) \Pi_j^l}. 
\]

Then done! I have derived the sectoral gravity model.

\section{Part 2 (Empirical): Estimating
gravity}\label{part-2-empirical-estimating-gravity}

\begin{enumerate}
\def\labelenumi{\arabic{enumi}.}
\item
  I have run the code from the start to line 583. I will personally talk
  to Yoto if I have any questions. I will also separately submit my code
  script.
\item
  I filled the missing parts of the code. The table below shows the
  results. For the first two columns, you can see that the coefficient
  estimates are positive for both RTA and WTO indicators. This is
  natural as the trade agreement would have led to lower trade costs.
  For the second column, you can also see that the coefficient estimate
  becomes smaller for RTA once you add the WTO indicator. This is
  probably because there was an omitted variable bias in the first
  column. As the WTO indicator is correlated with the RTA indicator,
  adding it will give us a more unbiased estimate. An interesting thing
  to note is that these coefficient estimates all become small in
  magnitude and insignificant once we do not account for domestic trade
  flows. This is probably related to the missing WTO effects puzzle we
  learned in class. In order to accurately estimate the WTO effects, we
  need to account for domestic trade flows (following the structural
  gravity model). For example, there might be some factors like
  trade-diversion effects of policies that are not accounted for in the
  third column.
\end{enumerate}

\begin{table}[!ht]
\centering
{
\def\sym#1{\ifmmode^{#1}\else\(^{#1}\)\fi}
\begin{tabular}{l*{3}{c}}
\toprule
                &\multicolumn{1}{c}{(1)}         &\multicolumn{1}{c}{(2)}         &\multicolumn{1}{c}{(3)}         \\
                & Trade         & Trade         & Trade         \\
\midrule
RTA member      &     0.26\sym{***}&     0.20\sym{***}&    -0.05         \\
                &   (0.07)         &   (0.07)         &   (0.06)         \\
WTO member      &                  &     0.48\sym{***}&    -0.04         \\
                &                  &   (0.08)         &   (0.17)         \\
\midrule
Observations    & 28482.00         & 28482.00         & 28068.00         \\
\bottomrule
\multicolumn{4}{l}{\footnotesize Standard errors in parentheses}\\
\multicolumn{4}{l}{\footnotesize \sym{*} \(p<0.10\), \sym{**} \(p<0.05\), \sym{***} \(p<0.01\)}\\
\end{tabular}
}

\end{table}

The next tables show the results for the NAFTA effect. Overall, you can
see that the coefficients are all positive and significant. The
advantage of creating separate dummies for different pairs of NAFTA
countries is that it gives us heterogeneous effects across different
pairs of countries.

\begin{table}[!ht]
\centering
{
\def\sym#1{\ifmmode^{#1}\else\(^{#1}\)\fi}
\begin{tabular}{l*{5}{c}}
\toprule
                &\multicolumn{1}{c}{(1)}         &\multicolumn{1}{c}{(2)}         &\multicolumn{1}{c}{(3)}         &\multicolumn{1}{c}{(4)}         &\multicolumn{1}{c}{(5)}         \\
                &   Trade         &   Trade         &   Trade         &   Trade         &   Trade         \\
\midrule
RTA member      &     0.14\sym{**} &                  &                  &                  &                  \\
                &   (0.07)         &                  &                  &                  &                  \\
WTO member      &     0.49\sym{***}&     0.49\sym{***}&     0.50\sym{***}&     0.50\sym{***}&     0.49\sym{***}\\
                &   (0.07)         &   (0.07)         &   (0.07)         &   (0.07)         &   (0.07)         \\
RTA exclude NAFTA&                  &     0.14\sym{**} &     0.11\sym{*}  &     0.11\sym{*}  &     0.11\sym{*}  \\
                &                  &   (0.07)         &   (0.06)         &   (0.06)         &   (0.07)         \\
NAFTA (CAN-USA) &                  &                  &     0.31\sym{***}&                  &                  \\
                &                  &                  &   (0.05)         &                  &                  \\
NAFTA (CAN-MEX) &                  &                  &     1.19\sym{***}&                  &                  \\
                &                  &                  &   (0.08)         &                  &                  \\
NAFTA (USA-MEX) &                  &                  &     0.76\sym{***}&                  &                  \\
                &                  &                  &   (0.08)         &                  &                  \\
NAFTA (CAN-USA) &                  &                  &                  &     0.44\sym{***}&     0.39\sym{***}\\
                &                  &                  &                  &   (0.07)         &   (0.09)         \\
NAFTA (USA-CAN) &                  &                  &                  &     0.16\sym{**} &     0.23\sym{***}\\
                &                  &                  &                  &   (0.06)         &   (0.08)         \\
NAFTA (CAN-MEX) &                  &                  &                  &     0.39         &     0.96\sym{***}\\
                &                  &                  &                  &   (0.35)         &   (0.10)         \\
NAFTA (MEX-CAN) &                  &                  &                  &     1.90\sym{***}&     1.46\sym{***}\\
                &                  &                  &                  &   (0.35)         &   (0.11)         \\
NAFTA (USA-MEX) &                  &                  &                  &     0.20\sym{***}&     0.25\sym{**} \\
                &                  &                  &                  &   (0.06)         &   (0.11)         \\
NAFTA (MEX-USA) &                  &                  &                  &     1.32\sym{***}&     1.27\sym{***}\\
                &                  &                  &                  &   (0.07)         &   (0.12)         \\
\midrule
Observations    & 28482.00         & 28482.00         & 28482.00         & 28482.00         & 28236.00         \\
\bottomrule
\multicolumn{6}{l}{\footnotesize Standard errors in parentheses}\\
\multicolumn{6}{l}{\footnotesize \sym{*} \(p<0.10\), \sym{**} \(p<0.05\), \sym{***} \(p<0.01\)}\\
\end{tabular}
}

\end{table}

\begin{enumerate}
\def\labelenumi{\arabic{enumi}.}
\setcounter{enumi}{2}
\tightlist
\item
  First, I ran the ETWFE using the jwdid estimator. The estimate you get
  from this regression is about -0.816 (se 0.114) for the sanction. This
  shows that the impact of sanctions on trade flows is very large and
  negative. I also plotted the leading, phasing in, and lift effects of
  the sanctions. I plot the event study plots below.
\end{enumerate}

\clearpage

\begin{figure}[H]

{\centering \pandocbounded{\includegraphics[keepaspectratio]{../figures/sanctions_eventstudy.png}}

}

\caption{Event study plots for the sanctions}

\end{figure}%

In the plot, you can see that there is already some downward trend in
the estimates before the sanctions. This might indicate some
anticipation effects of the sanctions. After the sanctions, the
estimates fall sharply and then start to recover. This is consistent
with the idea that the sanctions are effective in reducing trade flows.

\section{Part 3 (Empirical): GE
gravity}\label{part-3-empirical-ge-gravity}

\begin{enumerate}
\def\labelenumi{\arabic{enumi}.}
\item
  I have run the code from the start to finish. I will personally talk
  to Yoto if I have any questions. I will also separately submit my code
  script.
\item
  Optional.
\end{enumerate}

3 (a). If we add the Canada-US border, you can see that the coefficient
estimate is positive. Thus, relative to the baseline border, this
indicates that the border effect for Canada and the US is thinner than
other borders. But still, compared to the general border effect, this is
still not very large. Thus, while the border between Canada and the US
is thinner than other borders, the friction still matters.

\begin{table}[!ht]
\centering
{
\def\sym#1{\ifmmode^{#1}\else\(^{#1}\)\fi}
\begin{tabular}{l*{2}{c}}
\toprule
                &\multicolumn{1}{c}{(1)}         &\multicolumn{1}{c}{(2)}         \\
                &    Trade flow         &    Trade flow         \\
\midrule
log(distance)   &    -0.79\sym{***}&    -0.77\sym{***}\\
                &   (0.05)         &   (0.05)         \\
Contiguity      &     0.67\sym{***}&     0.63\sym{***}\\
                &   (0.11)         &   (0.11)         \\
Border          &    -2.47\sym{***}&    -2.53\sym{***}\\
                &   (0.12)         &   (0.12)         \\
USA-CAN border  &                  &     0.55\sym{***}\\
                &                  &   (0.14)         \\
\midrule
Observations    &  4761.00         &  4761.00         \\
\bottomrule
\multicolumn{3}{l}{\footnotesize Standard errors in parentheses}\\
\multicolumn{3}{l}{\footnotesize \sym{*} \(p<0.10\), \sym{**} \(p<0.05\), \sym{***} \(p<0.01\)}\\
\end{tabular}
}

\end{table}

3 (b). In the case of symmetric removal of the border, you can see the
gain from trade. This is especially large for Canada. While the USA also
has such a gain, it is relatively smaller and consumer prices actually
increase.

3 (c). In asymmetric removal, Canada's GDP rise becomes much smaller. It
also seems that the gain from trade is more concentrated on the producer
side.

\begin{longtable}[]{@{}
  >{\raggedright\arraybackslash}p{(\linewidth - 10\tabcolsep) * \real{0.1667}}
  >{\raggedright\arraybackslash}p{(\linewidth - 10\tabcolsep) * \real{0.1667}}
  >{\raggedright\arraybackslash}p{(\linewidth - 10\tabcolsep) * \real{0.1667}}
  >{\raggedright\arraybackslash}p{(\linewidth - 10\tabcolsep) * \real{0.1667}}
  >{\raggedright\arraybackslash}p{(\linewidth - 10\tabcolsep) * \real{0.1667}}
  >{\raggedright\arraybackslash}p{(\linewidth - 10\tabcolsep) * \real{0.1667}}@{}}
\toprule\noalign{}
\begin{minipage}[b]{\linewidth}\raggedright
Country
\end{minipage} & \begin{minipage}[b]{\linewidth}\raggedright
Cases
\end{minipage} & \begin{minipage}[b]{\linewidth}\raggedright
Change in Exports
\end{minipage} & \begin{minipage}[b]{\linewidth}\raggedright
Change in producer price
\end{minipage} & \begin{minipage}[b]{\linewidth}\raggedright
Change in consumer price
\end{minipage} & \begin{minipage}[b]{\linewidth}\raggedright
Change in GDP
\end{minipage} \\
\midrule\noalign{}
\endhead
\bottomrule\noalign{}
\endlastfoot
Canada & Symmetric & 98.18 & 31.13 & -21.74 & 67.56 \\
USA & Symmetric & 60.12 & 2.60 & 0.54 & 2.05 \\
Mexico & Symmetric & -1.05 & 0.52 & 1.06 & -0.53 \\
Canada & Asymmetric & 73.27 & 30.73 & 6.55 & 22.69 \\
USA & Asymmetric & 26.81 & -1.37 & -2.29 & 0.94 \\
\end{longtable}

4 (a). Suppose Trump imposes a 60\% tariff on all imports from China.
The change in welfare is as follows:

\begin{longtable}[]{@{}
  >{\raggedright\arraybackslash}p{(\linewidth - 10\tabcolsep) * \real{0.1667}}
  >{\raggedright\arraybackslash}p{(\linewidth - 10\tabcolsep) * \real{0.1667}}
  >{\raggedright\arraybackslash}p{(\linewidth - 10\tabcolsep) * \real{0.1667}}
  >{\raggedright\arraybackslash}p{(\linewidth - 10\tabcolsep) * \real{0.1667}}
  >{\raggedright\arraybackslash}p{(\linewidth - 10\tabcolsep) * \real{0.1667}}
  >{\raggedright\arraybackslash}p{(\linewidth - 10\tabcolsep) * \real{0.1667}}@{}}
\toprule\noalign{}
\begin{minipage}[b]{\linewidth}\raggedright
Country
\end{minipage} & \begin{minipage}[b]{\linewidth}\raggedright
Change in Exports
\end{minipage} & \begin{minipage}[b]{\linewidth}\raggedright
Change in producer price
\end{minipage} & \begin{minipage}[b]{\linewidth}\raggedright
Change in consumer price
\end{minipage} & \begin{minipage}[b]{\linewidth}\raggedright
Change in GDP
\end{minipage} & \begin{minipage}[b]{\linewidth}\raggedright
\end{minipage} \\
\midrule\noalign{}
\endhead
\bottomrule\noalign{}
\endlastfoot
USA & -4.83 & 1.27 & 1.51 & -0.23 & \\
China & -6.19 & -1.88 & -1.74 & -0.14 & \\
TZA & 0.06 & 0.03 & -0.31 & 0.34 & \\
NPL & -0.35 & -0.30 & -0.81 & 0.52 & \\
\end{longtable}

You can see that in terms of overall welfare, the USA actually loses
from the tariff. China also loses from the tariff, though not as much as
the USA. Tanzania and Nepal gain from the tariff. Perhaps it is due to
changes in exports and trade diversion effects of the tariff.

4 (b). Now suppose China retaliates. Then the change in welfare is as
follows:

\begin{longtable}[]{@{}
  >{\raggedright\arraybackslash}p{(\linewidth - 10\tabcolsep) * \real{0.1667}}
  >{\raggedright\arraybackslash}p{(\linewidth - 10\tabcolsep) * \real{0.1667}}
  >{\raggedright\arraybackslash}p{(\linewidth - 10\tabcolsep) * \real{0.1667}}
  >{\raggedright\arraybackslash}p{(\linewidth - 10\tabcolsep) * \real{0.1667}}
  >{\raggedright\arraybackslash}p{(\linewidth - 10\tabcolsep) * \real{0.1667}}
  >{\raggedright\arraybackslash}p{(\linewidth - 10\tabcolsep) * \real{0.1667}}@{}}
\toprule\noalign{}
\begin{minipage}[b]{\linewidth}\raggedright
Country
\end{minipage} & \begin{minipage}[b]{\linewidth}\raggedright
Change in Exports
\end{minipage} & \begin{minipage}[b]{\linewidth}\raggedright
Change in producer price
\end{minipage} & \begin{minipage}[b]{\linewidth}\raggedright
Change in consumer price
\end{minipage} & \begin{minipage}[b]{\linewidth}\raggedright
Change in GDP
\end{minipage} & \begin{minipage}[b]{\linewidth}\raggedright
\end{minipage} \\
\midrule\noalign{}
\endhead
\bottomrule\noalign{}
\endlastfoot
USA & -7.51 & 0.99 & 1.28 & -0.29 & \\
China & -7.11 & -1.58 & -1.31 & -0.27 & \\
TZA & 0.06 & 0.03 & -0.26 & 0.29 & \\
NPL & -0.27 & -0.23 & -0.67 & 0.44 & \\
\end{longtable}

In this case, the USA and China both lose more due to the retaliation.
Tanzania and Nepal still gain from the tariff, but the gain is smaller.

4 (c). Now suppose the world taxes US. Then the change in welfare is as
follows:

\begin{longtable}[]{@{}
  >{\raggedright\arraybackslash}p{(\linewidth - 10\tabcolsep) * \real{0.1667}}
  >{\raggedright\arraybackslash}p{(\linewidth - 10\tabcolsep) * \real{0.1667}}
  >{\raggedright\arraybackslash}p{(\linewidth - 10\tabcolsep) * \real{0.1667}}
  >{\raggedright\arraybackslash}p{(\linewidth - 10\tabcolsep) * \real{0.1667}}
  >{\raggedright\arraybackslash}p{(\linewidth - 10\tabcolsep) * \real{0.1667}}
  >{\raggedright\arraybackslash}p{(\linewidth - 10\tabcolsep) * \real{0.1667}}@{}}
\toprule\noalign{}
\begin{minipage}[b]{\linewidth}\raggedright
Country
\end{minipage} & \begin{minipage}[b]{\linewidth}\raggedright
Change in Exports
\end{minipage} & \begin{minipage}[b]{\linewidth}\raggedright
Change in producer price
\end{minipage} & \begin{minipage}[b]{\linewidth}\raggedright
Change in consumer price
\end{minipage} & \begin{minipage}[b]{\linewidth}\raggedright
Change in GDP
\end{minipage} & \begin{minipage}[b]{\linewidth}\raggedright
\end{minipage} \\
\midrule\noalign{}
\endhead
\bottomrule\noalign{}
\endlastfoot
USA & 12.82 & -2.43 & 7.89 & -9.56 & \\
China & 5.50 & -0.17 & -0.09 & -0.08 & \\
MEX & 20.64 & 0 & 1.28 & -1.28 & \\
\end{longtable}

For this case, I had to use other friend's result because somehow my
code was not converging, even though I seemed to have similar code to my
friend's. You can again see that USA suffers from the world tax. Rest of
the world also suffers from the worldwide tax.

4 (d). Finally, let's do the domestic tariff case. Then the change in
welfare is as follows:

\begin{longtable}[]{@{}
  >{\raggedright\arraybackslash}p{(\linewidth - 10\tabcolsep) * \real{0.1667}}
  >{\raggedright\arraybackslash}p{(\linewidth - 10\tabcolsep) * \real{0.1667}}
  >{\raggedright\arraybackslash}p{(\linewidth - 10\tabcolsep) * \real{0.1667}}
  >{\raggedright\arraybackslash}p{(\linewidth - 10\tabcolsep) * \real{0.1667}}
  >{\raggedright\arraybackslash}p{(\linewidth - 10\tabcolsep) * \real{0.1667}}
  >{\raggedright\arraybackslash}p{(\linewidth - 10\tabcolsep) * \real{0.1667}}@{}}
\toprule\noalign{}
\begin{minipage}[b]{\linewidth}\raggedright
Country
\end{minipage} & \begin{minipage}[b]{\linewidth}\raggedright
Change in Exports
\end{minipage} & \begin{minipage}[b]{\linewidth}\raggedright
Change in producer price
\end{minipage} & \begin{minipage}[b]{\linewidth}\raggedright
Change in consumer price
\end{minipage} & \begin{minipage}[b]{\linewidth}\raggedright
Change in GDP
\end{minipage} & \begin{minipage}[b]{\linewidth}\raggedright
\end{minipage} \\
\midrule\noalign{}
\endhead
\bottomrule\noalign{}
\endlastfoot
USA & -28.38 & -1.47 & 9.19 & -9.76 & \\
China & -6.16 & -1.51 & -1.25 & -0.27 & \\
TZA & -0.13 & -0.16 & 0.17 & -0.33 & \\
NPL & -0.34 & -0.39 & -0.41 & 0.02 & \\
\end{longtable}

You can see that many countries suffer from the domestic tariff. This is
because the domestic tariff is a tax on domestic consumers and this will
also propagate to other countries through trade diversion effects.
Interestingly, some countries like Nepal actually gain from the US's
domestic tariff.

4 (e). For the 60\% tariff on China and possible retaliation from China,
the estimates from the policy brief are pretty close to the results I
got. For example, they expect about -0.19\% to -0.12\% change in real
GDP, which is pretty close to my number of -0.23\% (though my code
expects a bigger loss for the USA). For retaliation, the policy brief
expected about -0.43\% to -0.21\% while my code expects about -0.29\%,
which is within the range. Thus, the model seems to do a pretty good job
in predicting the welfare effects of the tariff (though there are some
differences in the magnitude). On the other hand, the case for a
universal tariff on the US seems to be a bit off.

\begin{enumerate}
\def\labelenumi{\arabic{enumi}.}
\setcounter{enumi}{4}
\tightlist
\item
  Optional.
\end{enumerate}




\end{document}
